\RequirePackage{snapshot}
%%%%%%%%%%%%%%%%%%%%%%%%%%%%%%%%%%%%%%%%%%%%%%%%%%
% Basic setup. Most papers should leave these options alone.
\documentclass[fleqn,usenatbib]{mnras}

% MNRAS is set in Times font. If you don't have this installed (most LaTeX
% installations will be fine) or prefer the old Computer Modern fonts, comment
% out the following line
\usepackage{newtxtext,newtxmath}
% Depending on your LaTeX fonts installation, you might get better results with one of these:
%\usepackage{mathptmx}
%\usepackage{txfonts}

% Use vector fonts, so it zooms properly in on-screen viewing software
% Don't change these lines unless you know what you are doing
\usepackage[T1]{fontenc}

% Allow "Thomas van Noord" and "Simon de Laguarde" and alike to be sorted by "N" and "L" etc. in the bibliography.
% Write the name in the bibliography as "\VAN{Noord}{Van}{van} Noord, Thomas"
\DeclareRobustCommand{\VAN}[3]{#2}
\let\VANthebibliography\thebibliography
\def\thebibliography{\DeclareRobustCommand{\VAN}[3]{##3}\VANthebibliography}


%%%%% AUTHORS - PLACE YOUR OWN PACKAGES HERE %%%%%

% Only include extra packages if you really need them. Common packages are:
\usepackage{graphicx}	% Including figure files
\usepackage{amsmath}	% Advanced maths commands
\usepackage{fontawesome}

%%%%%%%%%%%%%%%%%%%%%%%%%%%%%%%%%%%%%%%%%%%%%%%%%%

%%%%% AUTHORS - PLACE YOUR OWN COMMANDS HERE %%%%%

\newcommand{\Msun}{M$_{\odot}$} % Msun (solar mass)
\newcommand{\microgauss}{$\mu$G} % microgauss
\newcommand{\vc}[1]{{\mathbf{#1}}}

% Workaround to fix the bug that prevents autoref from handling appendices properly
\newcommand{\aref}[1]{\hyperref[#1]{Appendix~\ref{#1}}}

% Autoref section names
\def\sectionautorefname{Section}
\def\subsectionautorefname{Section}
\def\subsubsectionautorefname{Section}
\def\paragraphautorefname{Section}
\def\figureautorefname{Figure}
\def\tableautorefname{Table}
\def\equationautorefname{equation}
\def\appendixautorefname{Appendix}


% MRK's editing macros
\usepackage{color}
\usepackage[normalem]{ulem}
\definecolor{darkgreen}{rgb}{0.13, 0.55, 0.13}
\newcommand{\red}[1]{{\textcolor{red}{#1}}}
\newcommand{\cyan}[1]{{\textcolor{cyan}{#1}}}
\newcommand{\mrkcut}[1]{{\red{\sout{#1}}}}
\newcommand{\mrkadd}[1]{{\cyan{#1}}}
\newcommand{\mrknote}[1]{{\textcolor{darkgreen}{[MRK: #1]}}}

%%%%%%%%%%%%%%%%%%%%%%%%%%%%%%%%%%%%%%%%%%%%%%%%%%

%%%%%%%%%%%%%%%%%%% TITLE PAGE %%%%%%%%%%%%%%%%%%%

% Title of the paper, and the short title which is used in the headers.
% Keep the title short and informative.
\title[Two-moment AMR radiation hydrodynamics on GPUs]{\textsc{Quokka}: A new code for two-moment AMR radiation hydrodynamics on GPUs}

% The list of authors, and the short list which is used in the headers.
% If you need two or more lines of authors, add an extra line using \newauthor
\author[B. D. Wibking et al.]{
    Benjamin D. Wibking$^{1}$\thanks{E-mail: ben.wibking@anu.edu.au (BDW)}
    and Mark R. Krumholz$^{1}$
\\
% List of institutions
$^{1}$Research School of Astronomy \& Astrophysics, Mount Stromlo Observatory, Cotter Road, Weston Creek, ACT 2611 Australia\\
}

% These dates will be filled out by the publisher
\date{Accepted XXX. Received YYY; in original form ZZZ}

% Enter the current year, for the copyright statements etc.
\pubyear{2021}

% Don't change these lines
\begin{document}
\label{firstpage}
\pagerange{\pageref{firstpage}--\pageref{lastpage}}
\maketitle

% Abstract of the paper
\begin{abstract}
    We present a new subcycling-in-time, block-structured adaptive mesh refinement (AMR) radiation hydrodynamics code for graphics processing units (GPUs). The equations of hydrodynamics are solved with the piecewise parabolic method (PPM) in a method-of-lines formulation and the radiative transfer is solved via the variable Eddington tensor (VET) radiation moment equations with a local closure. We combine explicit-in-time evolution of the radiation moment equations with the reduced speed-of-light approximation. We show results for a wide range of test problems for hydrodynamics, radiation, and coupled radiation hydrodynamics. On uniform grids in 3D, we achieve a peak of $69$ million hydrodynamic updates per second per GPU. For radiation hydrodynamics problems on uniform grids in 3D, our code scales from 4 GPUs to 256 GPUs with an efficiency of $>93$ per cent. The code is publicly released under an open-source license on \faGithub\href{https://github.com/BenWibking/quokka-code}{GitHub}.
\end{abstract}

% Select between one and six entries from the list of approved keywords.
% Don't make up new ones.
\begin{keywords}
radiation hydrodynamics -- numerical methods
\end{keywords}

%%%%%%%%%%%%%%%%%%%%%%%%%%%%%%%%%%%%%%%%%%%%%%%%%%

%%%%%%%%%%%%%%%%% BODY OF PAPER %%%%%%%%%%%%%%%%%%

\section{Introduction}
Detailed simulations of star formation require treating the dynamical effects of radiation produced by protostars and re-radiated by dust grains in the interstellar medium. A long-standing method to compute the radiation transport in hydrodynamic simulations is the approximation of flux-limited diffusion (FLD) \citep{Alme_1973}. This method evolves the radiation energy density under the assumption of a given closure for the radiation pressure tensor and the assumption that the time derivative of the radiation flux is zero.

A more accurate approximation is to evolve both the radiation energy density and the radiation flux, while still invoking a closure relation for the radiation pressure tensor. When the radiation pressure tensor (or rather, the ratio of the radiation pressure tensor to the radiatio energy density, known as the Eddington tensor) is computed via a formal solution of the angle-dependent radiative transfer equation, we obtain the quasidiffusion or variable Eddington tensor (VET) method \citep{Goldin_1964}. When retaining a local closure for the radiation pressure tensor in terms of the radiation energy density $E$ and the flux $F$, we obtain a local VET method, commonly referred to as the M1 (`moment-one`) method \citep{Minerbo_1978,Levermore_1984,Dubroca_1999,Gonzalez_2007}.

Something about the reduced speed of light (RSLA) method, citing \cite{Gnedin_2001} and \cite{Skinner_2013}\dots

% MRK I replaced all your reduced-speed-of-light's with RSLA, on the assumption that you will introduce this acronym at some point in the intro

\section{Methods}
\label{section:methods}
\subsection{Equations}
We solve the equations of radiation hydrodynamics \citep{Pomraning_1973,Mihalas_1984,Castor_2004} for an inviscid, nonrelativistic fluid
% MRK added note about LTE, since you're going to use the Planck function below
in local thermodynamic equilibrium in the so-called ``mixed frame,'' where the radiation variables are defined in an inertial frame (i.e., Eulerian simulation coordinates) and the radiation-matter interaction terms are written 
%MRK rephrased, because it is in fact possible to write the mixed-frame equations in a way that is exact, without series expanding in v/c -- there's a paper by Mihalas & Auer that does this. Of course the resulting expressions are terrifying, so no one uses them...
in the frame comoving with the fluid, with the transformations between the frames accounted for via the addition of radiation-matter exchange terms that depend explicitly on the ratio of fluid velocity to the speed of light, $\beta=v/c$.
We write the equations as follows:
\begin{align}
    \frac{\partial \rho}{\partial t} + \nabla \cdot (\rho \vc{v}) = 0 \, , \\
    \frac{\partial (\rho \vc{v})}{\partial t} + \nabla \cdot (\rho \vc{v} \vc{v} + \mathsf{P}) = \vc{G} \, , \\
    \frac{\partial E}{\partial t} + \nabla \cdot \left[(E + \mathsf{P})\vc{v}\right] = c G^0 \, , \\
    \frac{\partial E_r}{\partial t} + \nabla \cdot {\vc{F}_r} = -c G^0 \, , \\\
    \frac{1}{c^2}\frac{\partial \vc{F}_r}{\partial t} + \nabla \cdot \mathsf{P}_r = -\vc{G} \, ,
\end{align}
where $\rho$ is the gas density, $\vc{v}$ is the gas velocity, $E$ is the total energy density of the gas, $\mathsf{P} = \delta_{ij} P$ is the gas pressure tensor, $E_r$ is the radiation energy density, $F_r$ is the radiation flux, $\mathsf{P_r}$ is the radiation pressure tensor, $\nabla \cdot \rho \vc{v} \vc{v}$ denotes the sum $(\rho v_i v^j)_{,j}\,$, and $G^i$ is the radiation four-force, with $G_0$ the time-like component and $\vc{G}$ consisting of the space-like components. 
% MRK I expanded this to properly distinguish between flux mean and Planck mean, since your code does; also, rephrased to make the mixed versus comoving frame issue clear
In the mixed-frame formulation, the radiation four-force to order $\beta$ is
\begin{align}
-c G^0 = \rho (\kappa_P 4 \pi B - \kappa_E c E_r) + \rho \kappa_F \left( \frac{\vc{v}}{c} \cdot \vc{F}_r \right) \, , \\
-\vc{G} = -\rho \kappa_F \frac{\vc{F}_r}{c} + \rho \kappa_P \left(\frac{4 \pi B}{c}\right) \frac{\vc{v}}{c} + \rho \kappa_F \frac{\vc{v}\mathsf{P}_r}{c} \, ,
\end{align}
where
$\kappa_F$, $\kappa_E$, and $\kappa_P$ are the flux-mean, energy-mean, and Planck-mean specific opacities evaluated in the comoving frame,
$B$ is the Planck function evaluated at the gas temperature,
and $\vc{v} \mathsf{P}_r$ is the tensor contraction $v_j \mathsf{P}^{ij}$ \citep{Mihalas_1984}. The latter two terms in the expression for $\vc{G}$ correspond to the relativistic work term of \cite{Krumholz_2007} and are only important in the regime $\beta \tau \gtrsim 1$ (where $\tau$ is a characteristic optical depth), 
to which we cannot apply the RSLA (as discussed below),
so we neglect them. However, the term of order $\beta$ in the expression for $cG^0$ corresponds to the work done by the radiation force on the gas and can be the dominant term for problems of interest. 
% MRK I commented out the statement below because it is not correct. The whole point of the mixed-frame formulation is that kappa is the opacity evaluated in the comoving frame. We're not neglecting the difference between the comoving and lab-frame opacities; that difference is the reason that the various velocity-dependent terms in G appeared. 

To apply the RSLA to these equations, we first rewrite the radiation moment equations so that they have a factor of exactly $1/c$ next to each of the time derivatives:
\begin{align}
    \frac{1}{c} \frac{\partial E_r}{\partial t} + \nabla \cdot \left( \frac{\vc{F}_r}{c} \right) = -G^0 \, , \\\
    \frac{1}{c} \frac{\partial}{\partial t} \left( \frac{\vc{F_r}}{c} \right) + \nabla \cdot \mathsf{P_r} = -\vc{G} \, ,
\end{align}
then we replace this $1/c$ factor with a factor of $1/\hat c$, where $\hat c$ is the reduced speed of light, and multiply through by factors of $\hat c$ to obtain the conservation law form of the reduced speed of light radiation moment equations (e.g., \citealt{Skinner_2013}):
\begin{align}
    \frac{\partial E_r}{\partial t} + \nabla \cdot \left( \frac{\hat c}{c} \vc{F}_r \right) = -\hat c G^0 \, , \\\
    \frac{\partial \vc{F_r}}{\partial t} + \nabla \cdot (c \hat c \, \mathsf{P_r}) = -c \hat c \, \vc{G} \, .
\end{align}
The maximum wave speed of this system of equations is bounded by $\hat c$ (as long as the flux satisfies causality, i.e. $F_r \leq cE_r$). As emphasised by \cite{Skinner_2013}, all other factors of $c$ remain unchanged, and, since the factors of $c$ are unchanged on the right-hand side of the hydrodynamic equations, the reduced speed of light radiation hydrodynamic system does not conserve total energy or momentum for $\hat{c} \neq c$. 
When the left-hand side flux divergence terms are negligible, this nonconservation implies that the equilibrium temperature of the reduced speed of light system is slightly modified with respect to the correct equilibrium temperature, implying that we cannot apply the RSLA to problems in the equilibrium diffusion limit in general (see section \ref{test:exchange}).

Writing out the right-hand side terms explicitly, we obtain
\begin{align}
    \label{eq:hydro_continuity}
    \frac{\partial \rho}{\partial t} + \nabla \cdot (\rho \vc{v}) = 0 \, , \\
    \label{eq:hydro_momentum}
    \frac{\partial (\rho \vc{v})}{\partial t} + \nabla \cdot (\rho \vc{v} \vc{v} + \mathsf{P}) = \rho \kappa_F {\vc{F}_r / c} \, , \\
    \label{eq:hydro_energy}
    \frac{\partial E}{\partial t} + \nabla \cdot \left[(E + \mathsf{P})\vc{v}\right] = -c \rho (\kappa_P a_r T^4 - \kappa_E E_r) - \rho \kappa_F \left( \frac{\vc{v}}{c} \cdot \vc{F}_r \right) \, , \\
    \label{eq:rad_energy}
    \frac{\partial E_r}{\partial t} + \nabla \cdot \left( \frac{\hat c}{c} \vc{F}_r \right) = \hat c \rho \left(\kappa_P a_r T^4 - \kappa_E E_r \right) + \rho \kappa_F \left( \frac{\hat c}{c} \frac{\vc{v}}{c} \cdot \vc{F}_r \right) \, , \\\
    \label{eq:rad_flux}
    \frac{\partial \vc{F}_r}{\partial t} + \nabla \cdot (c \hat c \, \mathsf{P}_r) = -\hat c \rho \kappa_F \vc{F}_r \, .
\end{align}
These equations make no approximations about the frequency dependence of the radiation field. However, for computational tractability, in what follows we will approximate $\kappa_F$ with the Rosseland mean opacity $\kappa_R$, which yields the correct radiation force in the diffusion limit, and approximate $\kappa_E$ with the Planck mean opacity $\kappa_P$, which yields the correct energy absorption and emission in the optically-thin limit for fluids at rest \citep{Mihalas_1984}. However, we emphasise that the choice to set $\kappa_F \approx \kappa_R$ and $\kappa_E \approx \kappa_P$ is an additional approximation, and that others might be preferable depending on the physical system being simulated.  In future work, we plan to address the limitations of these approximate grey opacities via an extension of our method to the multigroup solution of the radiation moment equations.  Our present set of equations is sufficient for grey nonrelativistic radiation hydrodynamics in the semi-transparent regime, where we can neglect the `relativistic work term' that is important only in the dynamic diffusion ($\beta \tau \gtrsim 1$) regime, as described earlier.

\subsection{Hydrodynamics}
For the solution of the hydrodynamic subsystem (\autoref{eq:hydro_continuity}--\autoref{eq:hydro_energy}), we adopt a method-of-lines approach, discretizing the spatial variables while initially keeping the time variable continuous, thereby transforming the partial differential equations into a large set of ordinary differential equations. Then we integrate in time using a standard ordinary differential equation (ODE) integrator. The second-order strong stability preserving Runge-Kutta method (RK2-SSP; \citealt{Shu_1988}) is employed in our code. Such an approach has been successfully employed in several recent astrophysical hydrodynamics codes \citep{Skinner_2019,Stone_2020}. Somewhat surprisingly, we find that such a method-of-lines scheme is not stable when combining higher-order spatial reconstruction with forward Euler time integration, but is stable for timesteps $\mathcal{O}(\Delta x / (c_s + |v|))$ when used with higher-order Runge-Kutta methods. This is also found by \cite{Stone_2020}, who additionally find stable simulations are produced with a maximum stable timestep corresponding to the linear stability region of a given (second order or higher) Runge-Kutta integrator divided by the number of spatial dimensions.

For the spatial discretization, we use a standard finite volume method, where each spatial variable is represented by a volume average in each cell, on a rectangular, Cartesian grid. The grid spacing in each dimension may differ from each other, such that $\Delta x \neq \Delta y \neq \Delta z$ (where $\Delta x$ is the mesh spacing in the $x$ direction), but in our applications, we most often will want to choose $\Delta x = \Delta y = \Delta z$. When adaptive mesh refinement is enabled, the mesh spacing is additionally a function of refinement level $l$, where $(\Delta x)_{l} = (\Delta x_0) / 2^{l}$, where $\Delta x_0$ is the mesh spacing on the base (coarsest) grid. We enforce a resolution refinement factor of 2 from one level to the next in order to minimize numerical glitches arising from the discontinuous change in resolution, which can arise especially in problems where shocks cross the coarse-fine mesh interface at an oblique angle (e.g., \citealt{Fryxell_2000}).

We reconstruct the hydrodynamic variables on each face of each cell from the cell-average variables of the neighbouring cells. We perform this reconstruction using the piecewise parabolic method (PPM) \citep{Colella_1984} (hereafter, CW84) using the primitive hydrodynamic variables (density, velocity, and pressure). As is standard, the conversion from conserved (density, momentum, and energy) to primitive variables is carried out assuming that the volume average and cell centered states are equivalent, which is an approximation accurate to $\mathcal{O}(\Delta x^2)$. As noted by several authors, the PPM algorithm is therefore formally second-order accurate in spatial resolution. After the primitive variables have been defined, for the reconstruction step proper, the standard interface-centered PPM stencil is used:
\begin{align}
q_{j+1/2} = \frac{7}{12} (q_j + q_{j+1}) - \frac{1}{12} (q_{j+2} + q_{j-1}).
\end{align}
We follow the implementation of \cite{Stone_2020} in re-grouping the above terms symmetrically with respect to the interface ${i+{1/2}}$ so as to preserve exact symmetry in floating point arithmetic. On a mesh with uniform spacing between cells along the direction perpendicular to the reconstructed face, this stencil is fourth-order accurate.

The slope-limiting and contact steepening steps of CW84 are not performed. We enforce monotonicity of the reconstructed state by re-setting the interface state to lie between the values of the cells adjacent to the interface, following \cite{Mignone_2005}. This is followed by the extrema detection and overshoot correction step within each cell as described by CW84. In this step, the parabola assumed to exist across each cell is examined. If an `overshoot' (as defined by CW84) of the parabola is detected, we follow the original CW84 prescription of performing linear reconstruction on the side of the cell affected by the overshoot. If an extremum is instead detected, rather than forcing the reconstruction to a constant value across the cell as done by CW84, we revert to performing a linear reconstruction within the affected cell, following \cite{Balsara_2017}. Either of these steps may make the interface states discontinuous, with distinct states associated with each of the two cells adjacent to an interface. This reduces the order of accuracy of the stencil to either second order (when linear reconstruction is performed) or first order (when the monotonization is activated). Since the overall algorithm is a combination of second order and (usually) fourth order steps, the overall order of accuracy of PPM is often referred to as `third order' (e.g., \citealt{Stone_2020}).

Piecewise-linear (PLM) reconstruction based on the monotonized-central (MC) slope limiter is also implemented. PPM reconstruction is done by default, but PLM reconstruction is available to users via a compile-time option.

In some cases, especially in underresolved strong shocks, the previous steps do not provide sufficient dissipation to avoid oscillations. This problem was recognized by CW84, who proposed a shock flattening procedure in combination with a small amount of artificial viscosity. We fine that this shock flattening procedure is not sufficient in multidimensional problems. Instead, we follow \cite{Miller_2002}, who generalize the CW84 procedure for multidimensional hydrodynamics. Using this latter method, we find that no artificial viscosity is needed and we do not include any in our implementation.

Finally, in order to compute the flux of mass, momentum and energy between cells, we use the HLLC Riemann solver with the `primitive variable Riemann solver' wavespeeds and intermediate states \citep{Toro_2013}. We make the standard approximation that the face-average flux is the same as the face-centered flux, and therefore this step is also second-order accurate in spatial resolution. For each cell, the fluxes across each face are then added together to produce an unsplit spatial divergence term used by each stage of the Runge-Kutta integrator to advance the cell in time.

However, it has been long recognized that in strong grid-aligned shocks, the HLLC Riemann solver can unphysically amplify the so-called `carbuncle' instability \citep{Quirk_1994}. In astrophysical problems, this is most often encountered in strong explosions. Implementing additional dissipation in the form of artificial viscosity (e.g., \citealt{Gittings_2008}), the `H-correction' \citep{Sanders_1998}, or by adaptively switching to an HLL Riemann solver for computing fluxes perpendicular to strong shocks (e.g., \citealt{Quirk_1994,Skinner_2019}) are possible solutions to this issue. For future applications, we plan to implement an adaptive procedure to fix the carbuncle instability via the latter method.

\subsection{Radiation}
The radiation subsystem (\autoref{eq:rad_energy}--\autoref{eq:rad_flux}) is treated in a similar method-of-lines fashion. Our approach is most similar to that of \cite{Skinner_2019}, who evolve the radiation moment equations with a time-explicit method-of-lines approach but do not use a reduced speed of light nor a PPM reconstruction. In this work, we adopt the reduced speed of light radiation moment equations \citep{Skinner_2013} and evolve them explicitly in time with several radiation timesteps per hydrodynamic timestep. In the regime of applicability of the reduced-speed-of-light approximation, this approach allows a much more computationally efficient solution to the radiation moment equations, due to the fact that explicit methods have a greater arithmetic intensity per byte of data and do not require global communication across the computational domain in order to advance the solution in time. A high arithmetic intensity of the algorithm is greatly beneficial on GPUs, where the ratio of floating-point arithmetic performance to memory bandwidth is typically greater than on CPUs.

For the time integration on each AMR level, we evolve multiple radiation timesteps per hydrodynamic timestep using the RK2-SSP integrator \citep{Shu_1988}. We likewise use a finite volume representation of the radiation variables, with a PPM spatial reconstruction (or optionally, PLM) of the radiation energy density $E_r$ and reduced flux $\mathbf{f} = \mathbf{F_r} / cE_r$ carried out in the same fashion as described previously for the hydrodynamic variables, with the only exception that we do not employ a shock flattening procedure. There can exist unphysical radiation shocks when using local closures, since in general such closures make the radiation subsystem nonlinear, but there is no applicable shock flattening procedure to suppress this effect. Reconstruction is done in terms of the reduced flux $\mathbf{f}$ rather than the flux $\mathbf{F_r}$ in order to suppress unphysical fluxes $|\mathbf{F_r}| > cE$. This is effective in 1D problems, but in multidimensional problems, the magnitude of the radiation flux may still exceed $cE_r$, which is an unphysical state in which local closures cannot compute the Eddington factor at all. Reducing the order of reconstruction to first order (piecewise constant) when the interface states violate this constraint helps but does not eliminate the issue in all cases. For the purpose of computing the local closure only, we use rescale the flux such that $|\mathbf{F_r}| = cE_r$ whenever $|\mathbf{F_r}| > cE_r$.  For particularly difficult problems, especially in order to avoid unphysical instabilities in the propagation of non-grid-aligned optically-thin radiation fronts, we find that it is necessary to reconstruct the radiation variables using PLM reconstruction.

An issue with upwind finite volume methods for radiation transport is that in naive form, they do not give the correct behavior for diffusive regions where the optical depth per cell is much greater than 1 because the numerical diffusion dominates over the physical diffusion \citep{Lowrie_2001}. A common fix for this incorrect behavior is a modification of the Riemann solver in the optically thick regime \citep{Audit_2002,Skinner_2019,Mezzacappa_2020}, but this can lead to violations of the flux-limiting condition ($F_r \leq cE_r$) in the streaming regime \citep{Audit_2002}, especially at discontinuities in the opacity from optically-thin cells to optically-thick cells. The only apparent fix for this problem is to disable the optical-depth correction in the Riemann solver for those cells in which a flux-violating state is produced. We find that this fix is only activated when $f \rightarrow 1$, so it may not qualitatively affect the solution. However, we advocate refining on the gradient in the optical depth per cell in order to resolve the boundary layers in such situations whenever it is computationally feasible. In Appendix \ref{appendix:asymptotic_correction}, we show that such an optical-depth correction substantially improves the behavior of the method in the diffusive regime, but is not sufficient to obtain the correct diffusion solution in thermal waves.

For computing the flux of radiation quantities between cells, we use an HLL Riemann solver, with wavespeeds computed assuming the Eddington factors are fixed at the beginning of the timestep \citep{Balsara_1999}. This approach allows us to substitute different closure relations for the Eddington factors without requiring a modification of the Riemann solver, unlike previous implementations that are restricted to a single local closure (e.g., \citealt{Gonzalez_2007,Skinner_2013,Skinner_2019}). In principle, we could even use Eddington tensors computed via a short characteristics formal solution of the radiative transfer equation (e.g., \citealt{Davis_2012}), but we leave exploration of a non-local VET method to future work.

\subsubsection{Closure relations}
By default, we use the \cite{Levermore_1984} closure, which is derived by assuming that the radiation field is isotropic in some (unknown) reference frame and then computing a Lorentz transform from this reference frame to that of the given reduced flux $\mathbf{f}$. The radiation pressure tensor $\mathsf{P}$ produced by this closure can be expressed as (e.g., \citealt{Gonzalez_2007,Rosdahl_2013,Skinner_2013}):
\begin{align}
\mathsf{P} = \left( \frac{1 - \chi}{2} \mathsf{I} + \frac{3\chi - 1}{2} \mathbf{\hat n} \mathbf{\hat n} \right) E_r
\end{align}
where $\mathsf{I}$ is the identity tensor, and the Eddington factor $\chi$ and the flux direction cosine $\mathbf{\hat n}$ are
\begin{align}
\chi = \frac{3 + 4f^2}{5 + 2 \sqrt{4 - 3 f^2}} \, , \\
\mathbf{\hat n} = \frac{\mathbf{F_r}}{|F_r|} \, .
\end{align}
Other local closures have been derived based on other principles, e.g. \cite{Minerbo_1978,Levermore_1981}. There exists several comparisons of their performance with respect to angle-dependent transport solutions for both neutrino transport (e.g., \citealt{Janka_1992,Koerner_1992}) and radiation transport (e.g., \citealt{Olson_2000}). Because of its prominence in the neutrino transport literature as well as marginally favorable performance on some test problems, we also provide an implementation of the \cite{Minerbo_1978} closure. Any local closure is available to users given that they provide an implementation of a function that maps $f \rightarrow \chi$ for their preferred closure; no computational performance is sacrificed due to this flexibility.

\subsection{Matter-radiation coupling}
We use an implicit method to evaluate the source terms for both the radiation and hydrodynamic subsystems based on that of \cite{Howell_2003}\dots

\section{Test problems}
\label{section:tests}

\subsection{Hydrodynamics}
\subsubsection{Sound wave}
\subsubsection{Contact wave}
\subsubsection{Stationary shock tube}
\subsubsection{`Leblanc' test}
\subsubsection{Wave-shock interaction (Shu-Osher test)}
\subsubsection{Slow-moving shock}
\subsubsection{Strong rarefaction}
\subsubsection{Kelvin-Helmholtz problem}
\subsubsection{2D Implosion problem}

\subsection{Radiation}
\subsubsection{Marshak wave}
\subsubsection{Su-Olson problem}
\subsubsection{Radiation pressure tube}
\subsubsection{Radiation-matter energy exchange}
\subsubsection{Shadow test}
\subsubsection{Beam test}
\subsubsection{Optically-thin wind}
\subsubsection{Crooked-pipe problem}

\subsection{Radiation hydrodynamics}
\subsubsection{Subcritical radiative shock}
\subsubsection{Radiation-driven dust shell}

\section{Performance and scaling}
\label{section:performance}

\section{Discussion and Conclusions}
\label{section:discussion}
\subsection{Range of applicability}
\subsection{Planned applications}

The future is bright for radiation hydrodynamics on GPUs\dots

\section*{Acknowledgements}

This research was supported by the Australian Research Council through its Discovery Projects and Future Fellowship Funding Schemes, awards DP190101258 and FT180100375. This research was undertaken with the assistance of resources and services from the National Computational Infrastructure (NCI), which is supported by the Australian Government.

\emph{Software:} AMReX \citep{the_amrex_development_team_2021_5363443},
matplotlib \citep{Hunter:2007},
numpy \citep{harris2020array}.

%%%%%%%%%%%%%%%%%%%%%%%%%%%%%%%%%%%%%%%%%%%%%%%%%%
\section*{Data Availability}
The source code and entire commit history for \textsc{Quokka} is hosted in this public \faGithub\href{https://github.com/BenWibking/quokka-code}{GitHub repository}. The version of the source code used to produce the results in this paper as well as the output files at the final timestep for the simulations shown in the Figures are permanently archived at Zenodo DOI:XXX.

%%%%%%%%%%%%%%%%%%%% REFERENCES %%%%%%%%%%%%%%%%%%

% The best way to enter references is to use BibTeX:

\bibliographystyle{mnras}
\bibliography{quokka} % if your bibtex file is called example.bib

%%%%%%%%%%%%%%%%%%%%%%%%%%%%%%%%%%%%%%%%%%%%%%%%%%

%%%%%%%%%%%%%%%%% APPENDICES %%%%%%%%%%%%%%%%%%%%%

\appendix
\section{Asymptotic diffusion correction}
\label{appendix:asymptotic_correction}

%%%%%%%%%%%%%%%%%%%%%%%%%%%%%%%%%%%%%%%%%%%%%%%%%%


% Don't change these lines
\bsp	% typesetting comment
\label{lastpage}
\end{document}

% End of mnras_template.tex
